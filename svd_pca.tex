% arara: xelatex
\documentclass[12pt]{article}

% \usepackage{physics}

\usepackage{hyperref}
\hypersetup{
    colorlinks=true,
    linkcolor=blue,
    filecolor=magenta,      
    urlcolor=cyan,
    pdftitle={Overleaf Example},
    pdfpagemode=FullScreen,
    }

\usepackage{tikzducks}

\usepackage{tikz} % картинки в tikz
\usetikzlibrary{shapes, arrows, positioning}
\usepackage{microtype} % свешивание пунктуации

\usepackage{array} % для столбцов фиксированной ширины

\usepackage{indentfirst} % отступ в первом параграфе

\usepackage{sectsty} % для центрирования названий частей
\allsectionsfont{\centering}

\usepackage{amsmath, amsfonts, amssymb} % куча стандартных математических плюшек

\usepackage{comment}

\usepackage[top=2cm, left=1.2cm, right=1.2cm, bottom=2cm]{geometry} % размер текста на странице

\usepackage{lastpage} % чтобы узнать номер последней страницы

\usepackage{enumitem} % дополнительные плюшки для списков
%  например \begin{enumerate}[resume] позволяет продолжить нумерацию в новом списке
\usepackage{caption}

\usepackage{url} % to use \url{link to web}


\newcommand{\smallduck}{\begin{tikzpicture}[scale=0.3]
    \duck[
        cape=black,
        hat=black,
        mask=black
    ]
    \end{tikzpicture}}

\usepackage{fancyhdr} % весёлые колонтитулы
\pagestyle{fancy}
\lhead{}
\chead{SVD-PCA}
\rhead{}
\lfoot{}
\cfoot{}
\rfoot{}

\renewcommand{\headrulewidth}{0.4pt}
\renewcommand{\footrulewidth}{0.4pt}

\usepackage{tcolorbox} % рамочки!

\usepackage{todonotes} % для вставки в документ заметок о том, что осталось сделать
% \todo{Здесь надо коэффициенты исправить}
% \missingfigure{Здесь будет Последний день Помпеи}
% \listoftodos - печатает все поставленные \todo'шки


% более красивые таблицы
\usepackage{booktabs}
% заповеди из докупентации:
% 1. Не используйте вертикальные линни
% 2. Не используйте двойные линии
% 3. Единицы измерения - в шапку таблицы
% 4. Не сокращайте .1 вместо 0.1
% 5. Повторяющееся значение повторяйте, а не говорите "то же"


\setcounter{MaxMatrixCols}{20}
% by crazy default pmatrix supports only 10 cols :)


\usepackage{fontspec}
\usepackage{libertine}
\usepackage{polyglossia}

\setmainlanguage{russian}
\setotherlanguages{english}

% download "Linux Libertine" fonts:
% http://www.linuxlibertine.org/index.php?id=91&L=1
% \setmainfont{Linux Libertine O} % or Helvetica, Arial, Cambria
% why do we need \newfontfamily:
% http://tex.stackexchange.com/questions/91507/
% \newfontfamily{\cyrillicfonttt}{Linux Libertine O}

\AddEnumerateCounter{\asbuk}{\russian@alph}{щ} % для списков с русскими буквами
% \setlist[enumerate, 2]{label=\asbuk*),ref=\asbuk*}

%% эконометрические сокращения
\DeclareMathOperator{\Cov}{\mathbb{C}ov}
\DeclareMathOperator{\hCov}{\widehat{\mathbb{C}ov}}
\DeclareMathOperator{\Corr}{\mathbb{C}orr}
\DeclareMathOperator{\hCorr}{\widehat{\mathbb{C}orr}}
\DeclareMathOperator{\sCorr}{s\mathbb{C}orr}
\DeclareMathOperator{\Var}{\mathbb{V}ar}
\DeclareMathOperator{\col}{col}
\DeclareMathOperator{\row}{row}

\let\P\relax
\DeclareMathOperator{\P}{\mathbb{P}}

\DeclareMathOperator{\E}{\mathbb{E}}
\DeclareMathOperator{\tr}{trace}
\DeclareMathOperator{\trace}{trace}
\DeclareMathOperator{\Lagr}{Lagrange}

\DeclareMathOperator{\card}{card}

\DeclareMathOperator{\Convex}{Convex}
\DeclareMathOperator{\plim}{plim}

\usepackage{mathtools}
\DeclarePairedDelimiter{\norm}{\lVert}{\rVert}
\DeclarePairedDelimiter{\abs}{\lvert}{\rvert}
\DeclarePairedDelimiter{\scalp}{\langle}{\rangle}
\DeclarePairedDelimiter{\ceil}{\lceil}{\rceil}

\newcommand{\cN}{\mathcal{N}}
\newcommand{\cF}{\mathcal{F}}

\newcommand{\RR}{\mathbb{R}}
\newcommand{\NN}{\mathbb{N}}
\newcommand{\hb}{\hat{\beta}}
\newcommand{\dPois}{\mathrm{Pois}}


\newcommand{\SSR}{\mathrm{SS}^{\text{res}}}
\newcommand{\SSE}{\mathrm{SS}^{\text{expl}}}
\newcommand{\SST}{\mathrm{SST}}


\begin{document}
Let's start with SVD decomposition of $X = UDV^T$.
\[
\begin{pmatrix}
    x_{11} & x_{12} & \dots & x_{1k} \\
    x_{21} & x_{22} & \dots & x_{2k} \\
    x_{31} & x_{32} & \dots & x_{3k} \\
    \vdots & \vdots &    & \vdots \\
    x_{n1} & x_{n2} & \dots & x_{nk} \\
\end{pmatrix} =
\begin{pmatrix}
    u_{11} & u_{12} & \dots & u_{1n} \\
    u_{21} & u_{22} & \dots & u_{2n} \\
    u_{31} & u_{32} & \dots & u_{3n} \\
    \vdots & \vdots &    & \vdots \\
    u_{n1} & u_{n2} & \dots & u_{nn} \\
\end{pmatrix} \cdot 
\begin{pmatrix}
    d_{11} &  &     &  \\
     & d_{22} &     &  \\
     &  & \ddots    &  \\
     &  &   &    d_{kk} \\
     &  &  &     \\
     &  &  &    \\
     &  &  &     \\
\end{pmatrix} \cdot
\begin{pmatrix}
    v_{11} & v_{12} & \dots & v_{1k} \\
    v_{21} & v_{22} & \dots & v_{2k} \\
    \vdots & \vdots &    & \vdots \\
    v_{k1} & v_{k2} & \dots & v_{kk} \\
\end{pmatrix}^T
\]

Blank entries are zeros! The singular values on the diagonal of $D$ are positive and are sorted from largest to lowest, $d_{11} > d_{22} > \dots > d_{kk}$.

The columns of the matrix $P = UD = XV$ are called principal components, $p_j = d_{jj} u_j$.

Let $\hat D$ be the modified $D$ matrix, where we keep only the first $r$ positive elements on the diagonal, $d_{11}$, \dots, $d_{kk}$ 
and replace the other elements by $0$.

For $r = 2$ the approximation of the original $X$ matrix may be written as $\hat X = U \hat D V^T$,
\[
\begin{pmatrix}
    \hat x_{11} & \hat x_{12} & \dots & \hat x_{1k} \\
    \hat x_{21} & \hat x_{22} & \dots & \hat x_{2k} \\
    \hat x_{31} & \hat x_{32} & \dots & \hat x_{3k} \\
    \vdots & \vdots &    & \vdots \\
    \hat x_{n1} & \hat x_{n2} & \dots & \hat x_{nk} \\
\end{pmatrix} =
\begin{pmatrix}
    u_{11} & u_{12} & \dots & u_{1n} \\
    u_{21} & u_{22} & \dots & u_{2n} \\
    u_{31} & u_{32} & \dots & u_{3n} \\
    \vdots & \vdots &    & \vdots \\
    u_{n1} & u_{n2} & \dots & u_{nn} \\
\end{pmatrix} \cdot 
\begin{pmatrix}
    d_{11} &  &     &  \\
     & d_{22} &     &  \\
     &  &  \phantom{d_{33}} & \phantom{d_{33}}  \\
     &  &   &     \\
     &  &  &     \\
     &  &  &    \\
     &  &  &     \\
\end{pmatrix} \cdot
\begin{pmatrix}
    v_{11} & v_{12} & \dots & v_{1k} \\
    v_{21} & v_{22} & \dots & v_{2k} \\
    \vdots & \vdots &    & \vdots \\
    v_{k1} & v_{k2} & \dots & v_{kk} \\
\end{pmatrix}^T
\]
The approximation $\hat X = U\hat D V^T$ may be also written as $\hat X = U_* D_* V_*^T$,
\[
\begin{pmatrix}
    \hat x_{11} & \hat x_{12} & \dots & \hat x_{1k} \\
    \hat x_{21} & \hat x_{22} & \dots & \hat x_{2k} \\
    \hat x_{31} & \hat x_{32} & \dots & \hat x_{3k} \\
    \vdots & \vdots &    & \vdots \\
    \hat x_{n1} & \hat x_{n2} & \dots & \hat x_{nk} \\
\end{pmatrix} =
\begin{pmatrix}
    u_{11} & u_{12}  \\
    u_{21} & u_{22} \\
    u_{31} & u_{32} \\
    \vdots & \vdots \\
    u_{n1} & u_{n2} \\
\end{pmatrix} \cdot 
\begin{pmatrix}
    d_{11} &  \\
     & d_{22} \\
\end{pmatrix} \cdot
\begin{pmatrix}
    v_{11} & v_{12}  \\
    v_{21} & v_{22} \\
    \vdots & \vdots \\
    v_{k1} & v_{k2} \\
\end{pmatrix}^T
\]

The matrix $U_*$ consists of the leftmost $r$ columns of $U$.
The matrix $D_*$ consists of the top-left $r\times r$ corner of $D$.
The matrix $V_*^T$ consists of the top $r$ rows of $V^T$.


Errors of approximation are stored in the matrix:
\[
X - \hat X = UDV^T - U \hat D V^T = U(D - \hat D) V^T
\]


Let's calculate the quality of this approximation. 

We'll use two facts from linear algebra: $\trace(LR) = \trace(RL)$ and  $\trace(XX^T) = \trace(X^TX) = \sum x_{ij}^2$.

\[
\sum_{ij} x_{ij}^2 = \trace(X^TX) = \trace(VD^T U^TUDV^T ) = \trace(DD^T U^TU)= \trace(D^TD) = \sum_{i} d_{ii}^2
\]
By the same argument
\[
\sum_{ij} p_{ij}^2 = \trace(P^TP) = \trace(D^T U^T UD) =  \trace(D^TD) = \sum_{i} d_{ii}^2
\]

Recall that all vectors in the $X$ matrix are standardized.
Sample variance of a typical vector $x_j$ is given by $\sum_i (x_ij - 0)^2 / (n-1) = 1$.
Hence $\norm{x_j}^2 = \sum_i x_{ij}^2 = (n-1)$.

Hence,
\[
\sum_{ij} x_{ij}^2 = \norm{x_1}^2 + \norm{x_2}^2 + \dots + \norm{x_k}^2 = k (n-1) = \sum_{i=1}^k d_{ii}^2;
\]
Or, 
\[
\sum_{ij} p_{ij}^2 = \norm{p_1}^2 + \norm{p_2}^2 + \dots + \norm{p_k}^2 = k (n-1) = \sum_{i=1}^k d_{ii}^2;
\]


The sum of squares of all approximation errors:
\[
Q = \sum_{ij} (x_{ij} - \hat x_{ij})^2 = \trace((X - \hat X)(X - \hat X)^T)
\]

What is inside the trace?
\[
(X - \hat X)(X - \hat X)^T = V(D - \hat D)^T U^T U(D - \hat D) V^T= V(D - \hat D)^T (D - \hat D) V^T
\]

Let's finalize the calculation of total approximation error $Q$:
\begin{multline*}
Q = \trace((X - \hat X)(X - \hat X)^T) = \trace(V(D - \hat D)^T (D - \hat D) V^T) = \trace((D - \hat D)^T (D - \hat D) V^TV) = \\
 = \trace((D - \hat D)^T (D - \hat D)) = \sum_{i=r+1}^{k} d_{ii}^2 = k (n-1) - \sum_{i=1}^r d_{ii}^2;
\end{multline*}


Let's regress all the columns of $X$ onto predictors $U_*$ and obtain predictions.

We use the standard formula $\hat y = X (X^T X)^{-1}X^T y$ with $U_*$ instead of $X$ and $X$ instead of $y$:
\[
    U_* (U_*^T U_*)^{-1} U_*^T X = U_* (I)^{-1} U_*^T U \hat D V^T = U_* D_* V_*^T = \hat X
\]

So! The $\hat X$ may be viewed as low-rank approximation of $X$ from SVD or as matrix of forecasts of regressions of every original regressor $x_j$ on $r$ first principal components $p_1$, \dots, $p_r$.
Hence the sum of all squared approximation errors $Q$ is the sum of all sum of squared residuals from these regressions
\[
Q = \sum_{ij} (x_{ij} - \hat x_{ij})^2 = \norm{x_1 - \hat x_1}^2 + \dots + \norm{x_k - \hat x_k}^2 = \SSR_1 + \dots + \SSR_k.
\]

Let's calculate the $R^2$ in the first regression:
\[
R^2_1 = 1 - \frac{\SSR_1 }{ \SST_1} = 1 - \frac{\norm{x_1 - \hat x_1}^2 }{ \norm{x_1}^2}= 1 - \frac{\norm{x_1 - \hat x_1}^2 }{ n - 1}
\]

The sum of all $R^2_j$ hence is
\[
\sum R_j^2 = k - \frac{\sum (x_{ij} - \hat x_{ij})^2}{n - 1}.
\]

The average $R^2_j$, also called proportion of variance explained, is
\[
\frac{\sum R_j^2}{k} = 1 - \frac{\sum (x_{ij} - \hat x_{ij})^2}{k(n - 1)} = 1 - \frac{Q}{k(n-1)} = \frac{\sum_{i=1}^r d_{ii}^2}{\sum_{i=1}^k d_{ii}^2}.
\]

How can we extract all these $R^2_j$ for different number of principal components?
For simplicity let's consider $x_1$ and the first two principal components.
The coefficient $R^2_1$ comes from regression of $x_1$ onto first $r = 2$ principal components $p_1$, $p_2$.
\[
\hat x_1 = v_{11} p_1 + v_{12} p_2.
\]
Artificial regressors $p_1$ and $p_2$ are orthogonal, hence $R^2_1$ may be decomposed into 
\[
R^2_1 = R^2_{11} + R^2_{12},
\]
where $R^2_{11}$ comes from the regression of $x_1$ onto $p_1$ and $R^2_{22}$ — from the regression of $x_1$ onto $p_2$.
The orthogonality of regressors also means that coefficient estimate $v_{11}$ may be obtained from a univariate regression of $x_1$ onto $p_1$ alone. 
Similarly the regression coefficient $v_{12}$ may be also obtained from a univariate regression of $x_1$ onto $p_2$ alone. 

In a simple regression of original variable $x_i$ onto principal component $p_j$ the $R^2_{ij}$ may be obtained as a squared value of sample correlation between $x_i$ and $p_j$.
The sample correlation matrix of all columns $X$ with all columns of $P$ is called loadings matrix, $L$.
The sample correlation is matrix is unaffected by scaling, so $L$ is also a sample correlation matrix of columns of $X$ and columns of $U$.
Sample correlation matrix is just a sample covariance of standardized vectors.
The columns in $X$ are already standardized so we need to standardize only columns of $P$, or columns of $U$.
\[
L = \begin{pmatrix}
    \sCorr(x_1, p_1) & \sCorr(x_1, p_2) & \ldots \\
    \sCorr(x_2, p_1) & \sCorr(x_2, p_2) & \ldots \\
    \vdots & \vdots & \vdots 
\end{pmatrix} = 
\frac{X^T P^{st}}{n - 1} = \frac{X^T U^{st}}{n - 1}
\]
As $P^{st} = U^{st} = \sqrt{n-1} U$ we get
\[
L =  \frac{X^T U^{st}}{n - 1} = \frac{VD^T U^T U\sqrt{n-1}}{n-1} = \frac{1}{\sqrt{n-1}}V D^T.
\]
Hence, in the row $1$ the sum of the first $r$ squared values will give the coefficient $R^2_1$,
\[
R^2_1 = \ell_{11}^2 + \ell_{12}^2 + \dots + \ell_{1r}^2.
\]
If we continue summation over all $k$ principal components we would obtain a perfect prediction of $x_1$,
\[
    \ell_{11}^2 + \ell_{12}^2 + \dots + \ell_{1k}^2 = 1.
\]
Hence, $\norm{\row_i L} = 1$.

Another meaning of loadings may be seen from other view of SVD:
\[
X = U D V^T = U (VD^T)^T = \sqrt{n - 1}U (VD^T / \sqrt{n - 1})^T = U^{st} L^T.
\]
Hence, if we decompose original regressor $x_i$ as a linear combination of standardized 
principal components then the weights will be given by the $i$-th row of the loadings matrix $L$.

The sample correlation matrix of columns of $X$ is
\[
\frac{X^T X}{n-1} = \frac{V D^T U^T UDV^T}{n - 1} = \frac{V D^T D V^T}{ n - 1} = V \Lambda V^T.
\]
Here $\Lambda$ is the diagonalized sample correlation matrix of columns of $X$.
On its diagonal we have $\lambda_i = d_{ii}^2 / (n - 1)$.
The eigenvectors of the sample correlation matrix are just columns of $V$. 

The loading matrix can be expressed in terms of $V$ and $\Lambda$:
\[
L = V \Lambda^{1/2} = V \cdot \begin{pmatrix}
    \sqrt{\lambda_1 } & 0 & \dots & 0 \\
    0  & \sqrt{\lambda_2} & \dots & 0 \\
    \dots & \dots & \dots & \dots \\
    0 & 0 & \dots & \sqrt{\lambda_k} \\
\end{pmatrix}.
\]
As $\lambda_i = d_{ii}^2 / (n-1)$ the average $R^2_j$ is also 
\[
\frac{\sum R_j^2}{k} =  \frac{\sum_{i=1}^r d_{ii}^2}{\sum_{i=1}^k d_{ii}^2} =  \frac{\sum_{i=1}^r \lambda_i}{\sum_{i=1}^k \lambda_i} .
\]

Two more facts about $L$ matrix:
\[
L^T L = \Lambda^{1/2} V^T V \Lambda^{1/2} = \Lambda^1 =  \begin{pmatrix}
    \lambda_1 & 0 & \dots & 0 \\
    0  & \lambda_2 & \dots & 0 \\
    \dots & \dots & \dots & \dots \\
    0 & 0 & \dots & \lambda_k \\
\end{pmatrix}
\]
Hence $\ell_i \perp \ell_j$ for $i \neq j$, $\norm{\ell_j}^2= \lambda_j = d_{jj} / (n - 1)$.
The sample correlation matrix of columns of $X$ may be written as
\[
LL^T = \frac{1}{n-1}V D^T D V^T = \frac{X^TX}{n - 1}
\]
We again see that $(LL^T)_{ii} = 1$ as sample variance of standardized $x_i$.

Let's create a new variable $q$ as a linear combination of columns of $X$ with weight $w$, $q = Xw$.

We would like to maximize the sample variance of $Xw$ while keeping $\norm{w} = 1$.
As sample variance for centered variable is just $\norm{Xw}^2 / (n - 1)$
this is equivalent
\[
\norm{Xw}^2  \to \max_{\norm{w} = 1}
\]

Solution 1 (no Lagrange, no matrix differential):

We can rewrite $\norm{Xw}^2$ as
\[
\norm{Xw}^2 = (Xw)^T Xw = w^T X^T Xw = w^T VD^T DV^T w
\]
Let's introduce $a = V^T w$.
Remark that $\norm{a}^2 = a^T a = w^T VV^T w = w^T w = 1$ and
\[
    \norm{Xw}^2 = a^T \cdot \begin{pmatrix}
        d_{11} &  & \dots &  \\
          & d_{22} & \dots &  \\
        \dots & \dots & \dots & \dots \\
         &  & \dots & d_{kk} \\
    \end{pmatrix} \cdot a
    = d_{11}^2 a_1^2 + \dots + d_{kk} a_k^2
\]
For $k = 3$ this may look like
\[
    \norm{Xw}^2 = 1.3 a_1^2 + 0.7 a_2^2 + 0.2 a_3^2.
\]
As $\norm{a}^2 = a_1^2 + \dots + a_k^2 = 1$ we conclude that the optimal point is $a = (1, 0, \dots, 0)$. 
And optimal $w$ is $w = Va = v_1$, the first column of $V$.

Solution 2 (Rayleigh quotient):

\[
\norm{Xw}^2  \to \max_{\norm{w} = 1}
\]

As $w^Tw = 1$,
\[
    \frac{\norm{Xw}^2}{n - 1} = \frac{w^T VD^T DV^T w}{n-1} = \frac{w^T VD^T DV^T w }{ (n - 1)w^Tw }   = \frac{w^T \Lambda w}{ w^T w}
\]
Now take differential 
\[
d \frac{w^T \Lambda w}{ w^T w} = \frac{...}{ (w^T w)^2}
\]

Solution 3 (with Lagrange)

Let's write the Lagrangian function wit Lagrange multiplier denoted $\mu$:
\[
\Lagr(w, \mu) = \norm{Xw}^2 + \mu (1 - \norm{w}^2).
\]
The differential with respect to $dw$ is 
\[
d \Lagr = 2w^T X^T X dw - 2 \mu w^T dw.
\]
First order conditions are
\[
\begin{cases}
2w^T X^T X - 2 \mu w^T = 0 \\
w^Tw = 1  
\end{cases}
\]
Or, $X^TX w = \mu w$. 
Hence $\mu$ should be the eigenvalue of $X^TX$ matrix and $w$ should be the eigenvector with $\norm{w} = 1$.
The optimal value of the function is 
\[
w^T X^TX w = w^T \mu w = \mu w^T w = \mu.
\]
We should take the highest eigenvalue, that is $\mu_1 = (n - 1)\lambda_1$ as $\lambda_i$ are eigenvalues of the matrix $X^TX/(n-1)$.


Some useful sources:

\url{https://www.math.chalmers.se/Stat/Grundutb/GU/MSA220/S19/extra/svd-related-theory.pdf}

\url{https://www.cs.columbia.edu/~djhsu/AML/lectures/notes-pca.pdf}

\end{document}

